\documentclass{article}
\usepackage{url}
\usepackage[margin = 1in]{geometry}
\usepackage{amsmath, amssymb, amsfonts, amsthm}
\author{Yilin Guo}
\title{Part 1 Proof}

\begin{document}
	\maketitle
	\section{Proof}
	 Provide a proof to derive the formulas for "SELECT AVG(X) FROM D WHERE c" query under "Fixed-size without Replacement" (i.e., row 3 and column 3) in Table 2 of the following paper: \url{http://web.eecs.umich.edu/~mozafari/php/data/uploads/approx_chapter.pdf}
	 
	 $\theta_c$ is the estimator of approximating $\overline{X}_c$ using $S$ (\texttt{AVG(X) FROM D WHERE C}), then $\theta_c$ equals the mean of sample tuples that satisfies condition, i.e. $\theta_c = \overline{Y_c}$.
	 
	 $W_k = \frac{\binom{N_c}{k}\binom{N - N_c}{n - k}}{\binom{N}{n}}$ is the probability that select $n$ samples $Y_1, Y_2, \dots, Y_n$ among which exactly $k$ samples $Y_{c1}, Y_{c2},\dots, Y_{ck}$ satisfying the condition. In the best case, there are at most $b = \min\{n, N_c\}$ samples to satisfy the condition. In the worst case, there are at least $a = \max\{1, n - (N - N_c)\} = \max\{1, n - N + N_c\}$ to satisfy the condition. Therefore, the expected value of the estimator could be calculated through the condition mean in $D$ times the total probability of selecting $n$ samples with possible satisfying conditions, i.e. $E[\theta_c] = \overline{X}_c\sum_{a}^{b}W_k = \overline{X_c}W$. If $N \leq N - N_c$, then $W \neq 1$, $E[\theta_c] - \overline{X}_c = \overline{X}_cW - X_c \neq 0$; in this case the estimator $\theta_c$ is biased. Otherwise, if $N < N - N_c$, then $W = 1$, $E[\theta_c] - \overline{X}_c = \overline{X}_cW - X_c = 0$; in this case the estimator $\theta_c$ is unbiased.
	 
	 
	 
	 
\end{document}